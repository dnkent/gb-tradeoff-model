% !BIB program = bibtex

\documentclass[12pt]{article}
\usepackage[left=1in,top=1in,right=1in,bottom=1in]{geometry}
\newcommand*{\authorfont}{\fontfamily{phv}\selectfont}

%%%%%%%%%%%%%%%%%
% Prep abstract %
%%%%%%%%%%%%%%%%%
\usepackage{abstract}
\renewcommand{\abstractname}{}    % clear the title
\renewcommand{\absnamepos}{empty} % originally center

\renewenvironment{abstract}
{{%
\setlength{\leftmargin}{0mm}
    \setlength{\rightmargin}{\leftmargin}%
}%
\relax}
{\endlist}

\makeatletter
\def\@maketitle{%
\newpage
%  \null
%  \vskip 2em%
%  \begin{center}%
\let \footnote \thanks
{\fontsize{18}{20}\selectfont\raggedright  \setlength{\parindent}{0pt} \@title \par}%
}
%\fi
\makeatother


\setcounter{secnumdepth}{0}


%%%%%%%%%%%%%%
% Prep Title %
%%%%%%%%%%%%%%

\title{Guns, Butter, and the International Costs of Anarchy} %\thanks{}  }

\date{}

\usepackage{titlesec}

%%%%%%%%%%%%%%%%
% Section Size %
%%%%%%%%%%%%%%%%

\titleformat*{\section}{\large\bfseries}
\titleformat*{\subsection}{\normalsize\bfseries}
\titleformat*{\subsubsection}{\normalsize\itshape}
\titleformat*{\paragraph}{\normalsize\itshape}
\titleformat*{\subparagraph}{\normalsize\itshape}


\newtheorem{hypothesis}{Hypothesis}

%%%%%%%%%%%%%%%%%%%%%%%
% Additional Packages %
%%%%%%%%%%%%%%%%%%%%%%%

\usepackage{tikz} % need for tikz figures
\usepackage{xunicode}


% Non xelatex font options -- bitstream charter
\usepackage[charter,cal=cmcal]{mathdesign}

\usepackage{setspace}
\setlength\parindent{24pt}

\usepackage{mathtools} % math
\usepackage[labelfont=bf]{caption}

\usepackage[round]{natbib}

%% Remove footnote indent
\usepackage[hang,flushmargin]{footmisc}

% add tightlist ----------
\providecommand{\tightlist}{%
\setlength{\itemsep}{0pt}\setlength{\parskip}{0pt}}

%%%%%%%%%%%%%%%%%%%
% And, we're off! %
%%%%%%%%%%%%%%%%%%%

\begin{document}

% \pagenumbering{arabic}% resets `page` counter to 1

{% \usefont{T1}{pnc}{m}{n}
\setlength{\parindent}{0pt}
\thispagestyle{plain}
{\fontsize{20}{22}\selectfont\raggedright
\maketitle  % title \par

}

%% Authors -- Enter Manually
{
\vskip 13.5pt\relax \normalsize
Daniel Kent\footnote{\href{mailto:kent.249@osu.edu}{\nolinkurl{kent.249@osu.edu}}}
\hskip 15pt \emph{\small \textit{The Ohio State University}}
}

}


\begin{abstract}

    \hbox{\vrule height .2pt width 39.14pc}

    \vskip 8.5pt % \small

\noindent How do states limit the severity of arms races? The security dilemma -- the means by which a states makes itself more secure leave others less secure -- is given pride of place in international relations theory. Indeed, even when a security dilemma-induced arms spiral does not precede war, such costs are seen as a mutually undesirable but inescapable fact of international anarchy. Though a sizeable literature has focused on certain variables which determine the severity of the security dilemma, the degree to which these factors vary in practice has been brought reasonably under question. Despite such pessimistic forecasts, if one examines trends in military spending among various enduring rivalries, then a simple realization follows: genuine arms races (back and forth military spending) are rare. I posit that arms races, even in a world where intentions are highly uncertain and capabilities fully fungible, are infrequent because most states simply cannot afford them. To demonstrate this argument I construct an agent-based model of military spending with a guns-butter tradeoff built-in.


\hbox{\vrule height .2pt width 39.14pc}

\end{abstract}


\vskip 6.5pt

\doublespacing

\section{Introduction}\label{introduction}

Does a country growing its military promote or deter war? According to the spiral model\footnote{This view s shared by defensive realism more broadly.}, building arms creates insecurity and promotes war. On the other hand, the deterrence model\footnote{This view instead is promoted by offensive realism.} cites a lack of military might as a cause of conflict in that it invites aggressive opportunistic states. According to these two prominent approaches to international relations theory, arming is both a cause of war and peace.

In comparing the two, evidence tends to support the spiral model over the deterrence model. Why?

Why is there less evidence for the spiral model than the deterrence model of conflict? Relatedly, given the relative lack of support for spirals as a cause of conflict, how do states limit the severity of arms races? The spiral model argues that states repeatedly arm in response to other states arming, creating escalating tension and then war. \citep[e.g.][]{downs1985, jervis1978, powell1993} On the other hand, the deterrence model argues that war occurs because states fail to arm in response to other states. \citep[e.g.][]{achen1989, huth1993} These two models are thought to be at odds with each other because one prescribes arming as a means of deterring conflict whereas the other cautions against arming due to the possibility of a war-inducing spiral. In comparing the two, \citet{braumoeller2008} finds that the deterrence model has been better associated with the outbreak of militarized disputes than the spiral model. Given the two models' seemingly contrary prescriptions for state behavior and their importance to IR theory, it follows to ask why the deterrence model better fits history and whether states have adopted certain strategies to limit the severity of spirals.


\singlespacing

\bibliographystyle{abbrvnat}
\bibliography{prospectus}

\end{document}